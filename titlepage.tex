\providecommand{\relativeRoot}{.}
\documentclass[\relativeRoot/ada.tex]{subfiles}

\begin{document}

\begin{titlepage}

\HRulebeta

\vspace{3cm}
  {\bfseries
	\Large\textbf{Adaptive Fractionation for Reducing Number of Fractions}
	}\\[14pt]
	{Master Thesis - Medical Physics}\\[1pt]
	\textsc{University of Zurich, Department of Physics}
	\\[5pt]
	
\begin{addmargin}[1cm]{1cm}


\begin{minipage}[t]{0.6\textwidth}
\emph{Author}\\
Janic Tom Weber\\\\
\emph{Research Group Leader}\\
Prof. Dr. Jan Unkelbach \\
\end{minipage}
\begin{minipage}[t]{0.34\textwidth}
\emph{Date}\\
March 02, 2023\\\\
\emph{Supervisors}\\
Dr. Roman Ludwig \\
Dr. Riccardo Dal Bello\\[10pt]
\end{minipage}

\end{addmargin}

	

	
\begin{abstract}

    Objective: Fractionated radiotherapy typically delivers the same dose in each fraction (Uniform Fractionation). Adaptive Fractionation is a technique proposed by Pérez Haas \cite{perezhaas_adaptive} to exploit inter-fractional motion by increasing the dose on days when the distance of tumor and dose-limiting OAR is large and decreasing the dose on days when the distance is small. For favourable patient geometries where distances are large, Adaptive Fractionation has shown to deliver small residual doses in final or close-to-final fractions. Developed is an extension of the Adaptive Fractionation model to minimise number of fractions used for a treatment, to prevent applying such small residual doses in the final or close-to-final fractions for favourable patient geometries and use the vacant treatment allocation slot for additional patients: On favourable days the dose is further increased to possibly finish the treatment in earlier fractions and on unfavourable days dose modification is conformed to standard Adaptive Fractionation utilising the prescribed maximum number of fractions. The extended concept is evaluated for patients with pancreas, adrenal glands and prostate tumors previously treated at the MR-Linac in $5$ fractions with ablative dose.

    Approach: Given daily adapted treatment plans, inter-fractional changes are quantified by sparing factors $\delta_t$ defined as the OAR-to-tumor dose ratio. The key problem of Adaptive Fractionation is to decide on the dose to deliver in fraction $t$, given $\delta_t$ and the dose delivered in previous fractions, but not knowing future $\delta_t$s. Optimal doses that minimise the expected biologically effective dose in the OAR BED$_{3}$ and the number of fractions, while delivering a minimal BED$_{10}$ tumor dose prescription, are computed using dynamic programming. Assumed is a normal distribution over $\delta$ with mean and variance estimated from previously observed patient-specific $\delta_t$s for modelling sparing factor distribution. Collected were data of $30$ patients from the MR-Linac treatment planning system for pancreas, adrenal glands and prostate cancer. The algorithm is evaluated retrospectively for two patients with pancreas tumor and one patient with tumor in the adrenal glands. In all three cases tumor dose was compromised due to proximity of bowel, stomach, or duodenum.

	Main Results: In two patients with pancreatic cancer reducing number of fractions with Adaptive Fractionation resulted in a BED$_3$ decrease of $16.7$ Gy respectively $2.8$ Gy compared to Uniform Fractionation. The treatment reduced the number of fractions to $2$ respectively to $4$ number of fractions. In one patient with cancer in adrenal glands, reducing number of fractions with Adaptive Fractionation led to no reduction in number of fractions and a BED$_3$ increase of $2.8$ Gy compared to Uniform Fractionation, due to an unfavourable planning sparing factor, that is used for estimating the mean of $\delta$.
	
\end{abstract}


\vfill 
\end{titlepage}


\end{document}